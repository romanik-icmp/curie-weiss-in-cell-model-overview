\section{Notation}

$\vb r$ - coordinate in three-dimensional space;

$r \equiv \abs{\vb r}$ - absolute value of $\vb r$.

$\gamma$ - configuration of particles.

$W_{N_v}(\gamma)$ - potential energy of the interparticle interaction.

$\Phi_{N_v}(\vb{r}_i, \vb{r}_j)$ - interaction potential between particles (the Curie-Weiss potential).

$J_1$ - characteristic energy for attraction.

$J_2$ - characteristic energy for repulsion.

$a=J_2/J_1$ - ration between attraction and repulsion characteristic energies.

$\Delta_{l}$ - a cubic cell.

$c$ - linear size of a cubic cell.

$v \equiv c^3$ - volume of a cubic cell.

$N_v$ - number of cells in the whole volume.

$I_{\Delta_l} (\vb{r})$ - indicator function.

$V$ - volume.

$N$ - number of particles.

$\rho$ - particle density.

$T$ - temperature.

$k_{\rm B}$ - Boltzmann constant.

$\beta$ - inverse temperature.

$\mu$ - chemical potential.

$z$ - activity.

$\Lambda$ - de Broglie thermal wavelength.

$\Xi$ - grand partition function.

$Z_N$ - configuration integral.

$\Omega$ - grand potential.

$P$ - pressure.

$S$ - entropy.

$\langle\ldots\rangle$ - average over the grand canonical distribution.

$\rho^* \equiv \frac{\langle N \rangle}{V}v$ - reduced particle density.

$T^* \equiv \frac{k_{\rm B}T}{J_1}$ - reduced temperature.

$p \equiv \beta J_1$ - reduced inverse temperature.

$P^* \equiv \frac{Pv}{J_1}$ - reduced pressure.

$\mu^* \equiv \frac{\mu}{J_1}$ - reduced chemical potential.

$S^* \equiv \frac{S}{k_{\rm B} \langle N \rangle}$ - reduced entropy.

$v^* \equiv \frac{v}{\Lambda^3}$ - dimensionless cell volume.

$K_n(T^*,\mu^*;y)$ - special functions.

$E(T^*,\mu^*;y)$, $E_1(T^*,\mu^*;y)$, $E_2(T^*,\mu^*;y)$ - quantities defined in~\eqref{def:reducedE}, \eqref{def:reducedE1}, \eqref{def:reducedE2}, respectively.

$\bar{y} = \bar{y}(T^*,\mu^*)$ - function that maximizes $E(T^*,\mu^*;y)$.



\section*{Abbreviations}

GPF - Grand Partition Function.