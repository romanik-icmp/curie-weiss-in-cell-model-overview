\section{Notation}

---

$V$ - volume;

$N_v$ - number of cubic cells in the whole volume $V$;

$\Delta_{l}$ - a cubic cell; the $l$-th cubic cell;

$c$ - linear size of a cubic cell.

$v \equiv c^3$ - volume of a cubic cell.

$\gamma$ - configuration of particles;

$\vb r$; $\vb{r}'$ - coordinate in three-dimensional space;

$r \equiv \abs{\vb r}$ - absolute value of $\vb r$;

$\vb{r}_i$ - space coordinate of the $i$-th particle;

$\vb{r}^N \equiv \vb{r}_1, ..., \vb{r}_N$ - space coordinates of particles in a configuration $\gamma$;

$N$; $\abs{\gamma}$ - number of particles in a configuration $\gamma$;

$\Phi_{N_v}(\vb{r}_i, \vb{r}_j)$ - interaction potential between particles (the Curie-Weiss potential);

$J_1$ - characteristic energy for attraction;

$J_2$ - characteristic energy for repulsion;

$a \equiv J_2/J_1$ - ratio between attraction and repulsion characteristic energies;

$W_{N_v}(\gamma)$; $W_{N_v}(\vb{r}^N)$ - potential energy of the interparticle interaction in $\gamma$;

$\mathbb{I}_{\Delta_l} (\vb{r})$ - indicator function for a cell $\Delta_{l}$;

$\gamma_l = \gamma \cap \Delta_l$ - a part of $\gamma$ contained in $\Delta_l$;

$N_l$; $\abs{\gamma_l}$ - number of particles in $\gamma_l$;

---

$\Xi$ - grand partition function;

$\zeta$ - activity;

$\beta$ - inverse temperature;

$k_{\rm B}$ - Boltzmann constant;

$T$ - temperature;

$\mu$ - chemical potential;

$\Lambda$ - de Broglie thermal wavelength;

$\hbar$ - Planck constant;

$m$ - mass of a particle;

$Z_N$ - configuration integral;

${\rm d} \vb{r}^N \equiv {\rm d}{\vb r_1} \dotsc {\rm d}{\vb r_N}$;

$\langle\ldots\rangle$ - average over the grand canonical distribution;

$\Omega = -k_{\rm B}T \ln \Xi$ - grand potential;

---

$T^* \equiv \frac{k_{\rm B}T}{J_1}$ - reduced temperature.

$\beta^* \equiv \beta J_1 = 1/T^*$; $p \equiv \beta J_1 = 1/T^*$ - reduced inverse temperature;

$\mu^* \equiv \frac{\mu}{J_1}$ - reduced chemical potential;

$P$ - pressure;

$P^* \equiv \frac{Pv}{J_1}$ - reduced pressure;

$\rho$ - particle density;

$\rho^* \equiv \frac{\langle N \rangle}{V}v$ - reduced particle density;

$S$ - entropy;

$S^* \equiv \frac{S}{k_{\rm B} \langle N \rangle}$ - reduced entropy;

$Q_c$ - critical value of a quantity $Q$ (may be any of listed above);

$v^* \equiv {v}/{\Lambda_c^3}$ - dimensionless cell volume;

---

$\delta_{nm}$ - Kronecker $\delta$-symbol;

$l$; $l'$ - index for a cubic cell;

$i$; $j$ - index for a particle;

$\mathbb{N}$ - set of natural numbers;

$\mathbb{N}_0$ - set of non-negative integer numbers;

$\mathbb{R}$ - set of real numbers;

$\mathbb{R}^{+}$ - set of positive real numbers;

$\varrho \in \mathbb{N}_0^{N_v}$; $\varrho = (\varrho_1, \varrho_2, \ldots, \varrho_{N_{v}})$ - vector with non-negative integer components;

$\nu(\gamma)=(N_1, N_2, \ldots, N_{N_v})$; $\nu(\gamma) = (\abs{\gamma_1}, \abs{\gamma_2}, \ldots, \abs{\gamma_{N_v}})$ - vector representing number of particles in each cell;

---

$f(n; \lambda)$ - non-normalized Poisson distribution;

$\lambda$ - parameter of the Poisson distribution;

$K_n(T^*,\mu^*;y)$ - special functions.

$E(T^*,\mu^*;y)$, $E_1(T^*,\mu^*;y)$, $E_2(T^*,\mu^*;y)$ - quantities defined in~\eqref{def:E}, \eqref{def:reducedE1}, \eqref{def:reducedE2}, respectively.

$\bar{y} = \bar{y}(T^*,\mu^*)$ - function that maximizes $E(T^*,\mu^*;y)$.



\section*{Abbreviations}
---

GPF - Grand Partition Function;

CW - Curie-Weiss

vdW - van der Waals